\chapter{The Large Hadron Collider}
    \chapterprecishere{
        ``Potentielle citation sans aucun rapport avec le sujet"\par\raggedleft--- \textup{Personne inconnue}, contexte à déterminer
    }
    
   
        
    \section{Introduction}
    
        The study of elementary particles naturally demands a stable source of particles. At the dawn of particle physics the two main sources were radioactive materials and the cosmic rays. However soon researchers became in need of a more reliable source of particles in terms of particle energy, luminosity and experimental repeatability. This has commenced the era of particle accelerators.\\
        The first examples of particle accelerators were designed in late 1920s and early 1930s. Two different designs emerged: linear and circular. The former accelerates particles via electric field during the single pass through the machine, while the latter uses magnetic field to make accelerated particles go in circles allowing to re-accelerate the same beam many times. On the other hand the circular design comprises energy losses due to Bremsstrahlung radiation.\\
        In the second half of the XX century the accelerators gradually got bigger and bigger in both size and center-of-mass energy of the accelerated particles. This has allowed to create an experimental basis for the development of modern particle physics, notably the Standard Model.\\
        Up to this day the biggest particle accelerator with the highest center-of-mass energy is the Large Hadron Collider (LHC). LHC is a circular collider that lies in a tunnel of 27 km under the French-Swiss border next to Geneva \cite{Bruening}. In 2012 two biggest experiments of LHC have claimed the discovery of the Higgs boson, the last elementary particle predicted by the Standard Model which was not yet discovered by that time. \cite{higgs_atlas}, \cite{higgs_cms}.
        
        \section{The principle of a circular collider}
        
        Circular colliders - how they work and why they are circular. Cyclotron frequency.
Pipes, dipoles, quadrupoles, RF cavities, interaction points.

        \section{The LHC acceleration sequence}
        A few sentences and pictures about the chain between a ballon of hydrogen and the collisions at the IPs. 

        \section{Bunching and luminosity }
How and why the beam is bunched. How the luminosity is delivered to the IP and why it is a very important observable

        \section{LHC performance during the low-mu run}
A few plots and tables containing info about the delivered and collected luminosity
