\chapter[DLAr TPC]{Chambre à projection temporelle à double phase d'argon liquide (DLAr TPC)}
\chapterprecishere{``Les oiseaux sifflent, le printemps siffle"\par\raggedleft--- \textup{Roi Burgonde,Kaamelott}}

    \lipsum[1]
    \newpage

    \section{Principe de base d'une DLAr TPC}
        \subsection{Une TPC simple}
            \subsubsection{Historique}
            \subsubsection{Reconstruction de trace en 3D}
            \subsubsection{Pourquoi l'argon?}
            \subsubsection{Limites et avantages}
            \subsubsection{Intérêt pour la physique des neutrinos}
        \subsection{L'amélioration "double phase" pour amplifier le signal}
        \subsection{les expériences passées de DLAr TPC}
        
    \section{Le détecteur lointain de DU$\nu$E}
        \subsection{Capacités requises}
        \subsection{Un problème de taille : nécessité d'une expérience plus petite} 
        
    \section{WA105 : un prototype de 300 tonnes de DLAr TPC}
        
\printbibliography