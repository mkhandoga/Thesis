\section{L'invariance de jauge en électromagnétisme}\label{sec::EM_classique}

\begin{wrapfigure}{r}{0.5\textwidth}
    \be\label{eq::Max_1}
        \Vec{\nabla}.\Vec{B}=0 \text{ (Pas de monopôles magnétiques)} 
    \ee 
    \be\label{eq::Max_2} 
        \frac{\partial \Vec{B}}{\partial t} + \Vec{\nabla}\times \Vec{E}=\Vec{0} \text{ (Loi de Faraday)}
    \ee 
    \be\label{eq::Max_3}
        \Vec{\nabla}.\Vec{E}=\frac{\rho}{\epsilon_0} \text{ (Loi de Gauss)}
    \ee
    \be\label{eq::Max_4}
        \frac{\partial \Vec{E}}{\partial t}-c^2 \Vec{\nabla}\times \Vec{B} =-\frac{\Vec{j}}{\epsilon_0} \text{ (Loi d'Ampère)}
    \ee
    \caption{Les quatre équations de Maxwell}
\end{wrapfigure}

Le fait que les très célèbres équations de Maxwell (\eqref{eq::Max_1} à \eqref{eq::Max_4}) soient invariante sous une certaine transformation nous a permit dans la section \ref{sec::gauges} de créer une théorie de la physique des particules à bases de symétries de jauge locale, qui se sont révélées extrêmement fructueuse. Nous revenons un peu plus ici sur cette transformation et sur comment elle découle des équations de Maxwell.

La première chose que l'on remarque à propos de ces équations, c'est que ce sont des équations différentielles couplées. Deux d'entre elles font intervenir les variations de $\Vec{E}$ et $\Vec{B}$ simultanément. Nous allons donc chercher à "découpler" ces équations pour trouver une expression pour $\Vec{E}(\Vec{r},t)$ et une expression pour $\Vec{B}(\Vec{r},t)$. Nous tomberons naturellement sur la jauge de Lorenz. 

Prenons la première équation \eqref{eq::Max_1}. Le cours de licence de calcule différentielle nous a appris que le gradient d'un rotationnel est toujours nul. Nous écrivons donc 
\be \label{eq::B}
    \Vec{B}=\Vec{\nabla}\times \Vec{A}
\ee
Où $\Vec{A}$ est un champ vectoriel quelconque. En mettant ce résultat dans \eqref{eq::Max_2} nous obtenons : 
\be\nonumber
    \frac{\partial \Vec{\nabla}\times \Vec{A}}{\partial t} + \Vec{\nabla}\times \Vec{E}=\Vec{0}
\ee 
Les opérateurs de dérivations temporels et spatiaux commutent, nous pouvons donc factoriser de la manière suivante : 
\be
    \Vec{\nabla}\times \left(\frac{\partial \Vec{A}}{\partial t} +\Vec{E}\right)=\Vec{0}
\ee 
De la même manière que le gradient d'un rotationnel fait zéro, le rotationnel d'un gradient fait aussi zéro. Nous pouvons tenter une astuce similaire avec :
\be \label{eq::E}
    \frac{\partial \Vec{A}}{\partial t} +\Vec{E} = -\Vec{\nabla}.\phi
\ee 
Où $\phi$ est un champ scalaire quelconque.

Si l'on tenait \textit{vraiment} à isoler $\Vec{E}$ et $\Vec{B}$, on utiliserait à présent les deux dernières équations de Maxwell pour avoir les équations du mouvement de $\Vec{A}$ et $\phi$. Mais ce n'est pas le propos ici, vous pouvez vous reporter à n'importe quel cours d'électromagnétisme pour cela (j'ai un faible pour le premier chapitre des cours de Bjorn Felsager\cite{Felsager}). Nous allons plutôt constater ce qu'il se passe si l'on modifie les champs $\Vec{A}$ et $\phi$ de la manière suivante : 
\beq \label{eq::gauge_transf}
    \phi(\Vec{r},t) \rightarrow \phi(\Vec{r},t)+\frac{\partial \chi(\Vec{r},t)}{\partial t}\\
    \Vec{A}(\Vec{r},t) \rightarrow \Vec{A}(\Vec{r},t)-\Vec{\nabla}.\chi(\Vec{r},t)
\eeq
Avec $\chi$ un champ scalaire quelconque \textbf{dépendant éventuellement de $\Vec{r}$ et $t$} (je les ai mise exceptionnellement ici pour souligner ce point, mais je les enlève à nouveau dans la suite pour ne pas alourdir les notations). Les champs $\Vec{E}$ et $\Vec{B}$ en sont modifiés ainsi (en reprenant les équations \eqref{eq::B} et \eqref{eq::E}) :

\beq 
    \Vec{B}=\Vec{\nabla}\times\left(\Vec{A}-\Vec{\nabla}.\chi\right) \nonumber \\
    \Rightarrow \Vec{B}=\Vec{\nabla}\times\Vec{A}
\eeq 
Puisque le rotationnel d'un gradient est nul. On retombe donc sur l'équation \eqref{eq::B}. Il en va de même pour $\Vec{E}$ et l'équation \eqref{eq::E} :
\beq
    \Vec{E}=-\frac{\partial \left(\Vec{A}-\Vec{\nabla}.\chi\right)}{\partial t} - \Vec{\nabla}.\left(\phi+\frac{\partial \chi}{\partial t}\right) \nonumber \\
    \Rightarrow \Vec{E}=-\frac{\partial \Vec{A}}{\partial t}-\frac{\partial \Vec{\nabla}.\chi}{\partial t} - \Vec{\nabla}.\phi+\Vec{\nabla}.\frac{\partial \chi}{\partial t} \nonumber \\
    \Rightarrow \Vec{E} = -\frac{\partial \Vec{A}}{\partial t} -\Vec{\nabla}.\phi
\eeq

Ceci signifie que nous pouvons modifier nos champs $\Vec{A}$ et $\phi$ avec n'importe quel champ $\chi$ qui nous arrange sans changer la physique (puisque nous ne \textit{mesurons} pas ces champs). Un choix particulier de $\chi$ s'appelle un choix de jauge, les champs $\Vec{A}$ et $\phi$ sont les champs de jauges, et la transformation \eqref{eq::gauge_transf} est appelée transformation de jauge. Chaque choix peu amener à un résultat exploitable différent, mais nous n'en discuterons pas plus ici.

Passons à présent à la notation relativiste. En prenant pour convention le tenseur métrique 
\be 
    \eta_{\mu\nu}=
    \begin{pmatrix}
        1 & 0 & 0 & 0 \\
        0 & -1 & 0 & 0 \\
        0 & 0 & -1 & 0 \\
        0 & 0 & 0 & -1
    \end{pmatrix}
\ee 
le champ de jauge de l'électromagnétisme s'écrit 
\be
    A^{\mu}=(\phi,\Vec{A})
\ee 
en covariant et 
\be\label{eq::gauge_field}
    A_{\mu}=(\phi,-\Vec{A})
\ee 
en contravariant. La transformation de jauge s'écrit alors 
\be \label{eq::gauge_transf_relat}
    A_{\mu} \rightarrow A_{\mu}+\partial_{\mu}\chi
\ee 
Et c'est de là que démarre la section \ref{sec::gauges}.