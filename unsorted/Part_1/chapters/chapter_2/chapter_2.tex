\chapter{Aspect expérimentaux}
\chapterprecishere{``Hé mon pâté..."\par\raggedleft--- \textup{Garçon égyptien}, Astérix et Obélix mission Cléopâtre}


    \lipsum[1]
    \newpage
    \section{Les grandes expériences de physique des neutrinos}
        \subsection{A quoi ressemble une expériences mesurant des neutrinos?}
            Les expériences de physique des neutrinos se divisent en plusieurs catégories, dépendant de ce que l'on cherche à mesurer (paramètres de la matrice PMNS, ordre des masses, phase de violation CP...) et de l'origine des neutrinos (accélérateurs, réacteurs nucléaires, solaire, supernovae, géologiques...) Elles partagent néanmoins toutes quelques caractéristiques due à la nature des interactions des neutrinos avec la matière.\\
            
            Les neutrinos et les antineutrinos n'interagissant que par interaction faible, il est nécessaire de compenser la section efficace d'interaction, très faible, par un grand volume fiduciel. C'est pourquoi les détecteurs de neutrinos sont les plus grands possibles\footnote{Il est intéressant de noter qu'il existe des expériences de neutrinos où ce n'est pas le neutrino qui est détecté, mais, par exemple, un électron émis en même temps lors d'une désintégration $\beta$. C'est le cas dans l'expérience KATRIN\cite{katrin} par exemple, qui cherche à mesurer directement la masse du neutrinos électronique en observant une légère déviation du spectre en énergie de l'électron en comparaison d'une masse nulle. Une telle expérience ne nécessite pas un volume gigantesque.}. Malgré cela le taux d'événement n'est pas élevé (quelques dizaines par jours attendu pour DU$\nu$E par exemple\cite{dune_rate}), un système de réduction de bruit de fond performant est donc requis. 
            \subsubsection{Long Baseline}
                histo L/E
            \subsubsection{Short Baseline}
                histo L/E
            \subsubsection{Télescopes}
            \subsubsection{Autres expériences}
        \subsection{les différents types de détecteurs}
            \subsubsection{Cerenkov}
            \subsubsection{TPC}
            \subsubsection{MIND}
        
        \subsection{Les expériences passées d'oscillation de neutrinos et leurs résultats}
        
        \subsection{Les expériences à venir et leurs objectifs}
        
        \begin{table}[h!]
            \centering
            \begin{tabular}{|c||c|}
                \hline
                1 pied & 1 autre pied \\
                \hline
                \hline
                1 troisième pied & et un dernier pied \\
                \hline
            \end{tabular}
            \caption{Une table à 4 pieds}
            \label{Tab::table}
        \end{table}
        La grande majorité des tables ont 4 pieds (voir table \ref{Tab::table}) comme l'explique Rutherford dans son article: \cite{Rutherford}.
        
    \section{Deep Underground Neutrino Experiment, alias DU\texorpdfstring{$\nu$}{Lg}E}
        \subsection{Où, quand, comment?}
        \subsection{Ses objectifs}
        \subsection{Ses détecteurs}

\printbibliography